%!TEX TS-program = xelatex
%!TEX encoding = UTF-8 Unicode
\documentclass[12pt, a4paper]{article} %extarticle

%%%%%%%%%%%% Пакеты %%%%%%%%%%%%%%%%%%
\usepackage{polyglossia} % языковой пакет

\usepackage{pdfpages} % пакет для импорта pdf-файлов

\usepackage{tocvsec2} %%%%%%%%%%%%%%%%%%%%%%%%%%%%%%%%%

\usepackage[labelsep=period]{caption}

\usepackage{caption}

%\usepackage{sectsty}
% нужен для более удобного форматирования, но т.к. мы используем
% 14 шрифт, то приходится использовать extrarticle, который
% не поддерживается.
% ладно.

\usepackage{graphicx} % пакет для использования графики (чтобы вставлять рисунки, фотографии и пр.)


% качественные листинги кода
\usepackage{listings}
\usepackage{lstfiracode}



\usepackage{amsmath} % поддержка математических символов

\usepackage{url} % поддержка url-ссылок

\usepackage[numbers]{natbib} % для нумерации
% \usepackage{natbib} % менеджер цитирования natlib.

\bibliographystyle{unsrtnat} % выбираем стиль библиографии отсюда: https://www.overleaf.com/learn/latex/Natbib_bibliography_styles

\setcitestyle{authoryear, open={(},close={)}} % Определяем стиль цитирования. Указываем, чтобы цитирование в тексте вставлялось в формате (Автор, год).

\usepackage{multirow} % таблицы с объединенными строками

\usepackage{hyperref} % пакет для интеграции гиперссылок

\usepackage{indentfirst} % пакет для отступа абзаца


\usepackage{chngcntr} % пакет подписей и нумерации рисунков

\usepackage{cancel} % пакет для перечеркивания формул



%%%%%%%%%%%%%%%%%%%%%%%%%%%%%%%%%%%%%%%%%%%%%%%%%%%%%%%%%%%%%%

\usepackage{titletoc}
\usepackage{amsmath}
\usepackage{float}
\usepackage{titlesec}
\usepackage{diagbox}
\usepackage{tikz}
\usepackage{enumitem}
\usetikzlibrary{automata, arrows.meta, positioning}
%%%%%%%%%%%% Формат %%%%%%%%%%%%%%%%%%
\include{Settings/format}

\makeatletter % Reference list option change
\renewcommand\@biblabel[1]{#1.} % from [1] to 1.
\makeatother

\titleformat{\section}
   {\normalfont\Large\bfseries}{Глава \thesection.}{0.5ex}{\centering}

\addto\captionsrussian{%
  \renewcommand{\contentsname}%
    {\centering Оглавление}%
}
\contentsmargin{0pt}
\dottedcontents{section}[2.3em]{}{2.3em}{5pt}
\dottedcontents{subsection}[5.5em]{}{3.2em}{5pt}


%%%%%%%%%%%% Начало документа %%%%%%%%%%%%%%%%%%
\begin{document}

% https://www.overleaf.com/learn/latex/Natbib_citation_styles
\setcitestyle{square,numbers,comma}


%\includepdf[pages=-]{titlepage.pdf}


%%%%%%%%%%%%%%%  Команды %%%%%%%%%%%%%%%%%%
% Список команд

%%%%%%%%%%%% \image %%%%%%%%%%%%%%%%%%

% \image {Имя изображения.расширение}{Подпись к рисунку}{Скейл Изображения}

\newcommand{\image}[3]{
\begin{figure}[!htb]
	\centering
	\includegraphics[width=#3\textwidth]{#1}
	\caption{#2}
\end{figure}
}

% Формула с нумерацией
\newcommand{\centreq}[1]{
\begin{equation}
#1
\end{equation}
}

% Главы с центрированием
\newcommand{\csection}[1]{
	\section{\centering#1}
}
\newcommand{\ocsection}[1]{
	\section*{\centering#1}
}
\newcommand{\csubsection}[1]{
	\subsection{\centering#1}
}
\newcommand{\ocsubsection}[1]{
	\subsection*{\centering#1}
}



%%%%%%%%%%%%%%%%%%%%%%%%%%%%%%%%%%%%%%


%%%%%%%%%%%% Содержание %%%%%%%%%%%%%%%%%%
\newpage
\setcounter{page}{3}
\tableofcontents
\newpage

%%%%%%%%%%%%%%%%%%%%%%%%%%%%%%%%%%%%%%%%%%%%%%%%%%%%%%%%%%%%%%%%%%%%
\section{Теоретическая часть.}
\csubsection{Конечные автоматы. Основные понятия и определения.}

Всем привет я проснулся :)

\newpage
\section{Заключение}

В данной курсовой работе была рассмотрена теория конечных автоматов, показаны примеры их применения, а также 
выполнена задача по поиску количества слов длины 3 заданного в виде таблицы переходов автомата. \\
В ходе выполнения задачи была проведена работа по построению грамматики языка, с помощью которой
далее были получены системы уравнения языка. Затем был осуществлен переход из системы уравнения в языках
в систему производящих функций с помощью гомоморфизма в ряды.

Система была решена относительно производящей функции $f_0$, благодаря чему было получено разложение 
в формальный степенной ряд (17). Все коэффициенты полученного ряда оказались существенными, также в ходе
проверки было обнаружено, что количество подходящих слов определенной длины и коэффициенты при одночленах
действительно совпадают, а значит, задача была полностью выполнена.


% Список приложений

\newpage
\renewcommand*\refname{Список литературы}
\nocite{*}
\bibliography{refs}
\addcontentsline{toc}{section}{Список литературы}
\newpage


\end{document}

% Шаблон для курсовой/диплома/отчета
% Сделано Stulk3
%https://github.com/Stulk3/Latex-Template-for-Report-Diploma-Thesis




\documentclass[14pt, a4paper]{extarticle}


%%%%%%%%%%%% Пакеты %%%%%%%%%%%%%%%%%%
\usepackage{polyglossia} % языковой пакет

\usepackage{pdfpages} % пакет для импорта pdf-файлов

\usepackage{tocvsec2} %%%%%%%%%%%%%%%%%%%%%%%%%%%%%%%%%

\usepackage[labelsep=period]{caption}

\usepackage{caption}

%\usepackage{sectsty}
% нужен для более удобного форматирования, но т.к. мы используем
% 14 шрифт, то приходится использовать extrarticle, который
% не поддерживается.
% ладно.

\usepackage{graphicx} % пакет для использования графики (чтобы вставлять рисунки, фотографии и пр.)


% качественные листинги кода
\usepackage{listings}
\usepackage{lstfiracode}



\usepackage{amsmath} % поддержка математических символов

\usepackage{url} % поддержка url-ссылок

\usepackage[numbers]{natbib} % для нумерации
% \usepackage{natbib} % менеджер цитирования natlib.

\bibliographystyle{unsrtnat} % выбираем стиль библиографии отсюда: https://www.overleaf.com/learn/latex/Natbib_bibliography_styles

\setcitestyle{authoryear, open={(},close={)}} % Определяем стиль цитирования. Указываем, чтобы цитирование в тексте вставлялось в формате (Автор, год).

\usepackage{multirow} % таблицы с объединенными строками

\usepackage{hyperref} % пакет для интеграции гиперссылок

\usepackage{indentfirst} % пакет для отступа абзаца


\usepackage{chngcntr} % пакет подписей и нумерации рисунков



%%%%%%%%%%%%%%%%%%%%%%%%%%%%%%%%%%%%%%%%%%%%%%%%%%%%%%%%%%%%%%

\usepackage{titletoc}
%%%%%%%%%%%%%%%%%%%%%%%%%%%%%%%%%%%%%%

%%%%%%%%%%%% Формат %%%%%%%%%%%%%%%%%%
%%%%%%%%%%%%%%%%% Оформление ГОСТА%%%%%%%%%%%%%%%%%

% Все параметры указаны в ГОСТЕ на 2021, а именно:

% Шрифт для курсовой Times New Roman, размер – 14 пт.
\setdefaultlanguage[spelling=modern]{russian}
    \setotherlanguage{english}
    
\setmonofont{Times New Roman}
\setmainfont{Times New Roman} 
\setromanfont{Times New Roman} 
\newfontfamily\cyrillicfont{Times New Roman}




% шрифт для URL-ссылок
\urlstyle{same} 

% Междустрочный интервал должен быть равен 1.5 сантиметра.
\linespread{1.5} % междустрочный интервал


% Каждая новая строка должна начинаться с отступа равного 1.25 сантиметра.
\setlength{\parindent}{1.25cm} % отступ для абзаца


% Текст, который является основным содержанием, должен быть выровнен по ширине по умолчанию включен из-за типа документа в main.tex

% Ширина левого поля должна равняться 3 сантиметра, а правое 1 сантиметра. Верхнее и нижнее должны равняться 2 сантиметра.
\usepackage[left=3cm,right=1.5cm,top=2cm,bottom=2cm]{geometry} % поля

%%%%%%%%%%%%%%%%%% Дополнения %%%%%%%%%%%%%%%%%%%%%%%%%%%%%%%%%

% Путь до папки с изображениями
\graphicspath{ {./Images/} }

% Внесение titlepage в учёт счётчика страниц
\makeatletter
\renewenvironment{titlepage} {
	\thispagestyle{empty}
}


% Цвет гиперссылок и цитирования
\usepackage{hyperref} 
 \hypersetup{ 
     colorlinks=true, 
     linkcolor=black, 
     filecolor=blue, 
     citecolor = black,       
     urlcolor=blue, 
     }
    

% Нумерация рисунков
\counterwithin{figure}{section}

% Нумерация таблиц
\counterwithin{table}{section}

\counterwithin{table}{section}

% шрифт для листингов с лигатурами
\setmonofont{FiraCode-Regular.otf}[
	SizeFeatures={Size=10},
	Path = Settings/,
	Contextuals=Alternate
]



% настройка подсветки кода и окружения для листингов
%\usemintedstyle{colorful} % делает подсветку для кода
\newenvironment{code}{\captionsetup{type=listing}}{}


% Посмотреть ещё стили можно тут https://www.overleaf.com/learn/latex/Code_Highlighting_with_minted
%\makeatletter
%\renewcommand*\l@section{\@dottedtocline{1}{1.5em}{1em}}
%\renewcommand*\l@subsection{\@dottedtocline{1}{3em}{1.5em}}
%\makeatother
\addto\captionsrussian{%
  \renewcommand{\contentsname}%
    {\centering Оглавление}%
}
\contentsmargin{0pt}
\dottedcontents{section}[2.3em]{}{2.3em}{5pt}
\dottedcontents{subsection}[5.5em]{}{3.2em}{5pt}
%%%%%%%%%%%%%%%%%%%%%%%%%%%%%%%%%%%%%%





%%%%%%%%%%%% Начало документа %%%%%%%%%%%%%%%%%%
\begin{document}

% https://www.overleaf.com/learn/latex/Natbib_citation_styles
\setcitestyle{square,numbers,comma}



%\includepdf[pages=-]{titlepage.pdf}

% Если нужно вставить свой титульный лист, то если загрузить его в формате .pdf и переназвать файл на titlepage, то он вставится в начало документа
%%%%%%%%%%%%%%%%%%%%%%%%%%%%%%%%%%%%%%%%%%%%%%%%%

%%%%%%%%%%%%%%%  Команды %%%%%%%%%%%%%%%%%%
% Список команд

%%%%%%%%%%%% \image %%%%%%%%%%%%%%%%%%

% \image {Имя изображения.расширение}{Подпись к рисунку}{Скейл Изображения}

\newcommand{\image}[3]{
\begin{figure}[!htb]
	\centering
	\includegraphics[width=#3\textwidth]{#1}
	\caption{#2}
\end{figure}
}

% Формула с нумерацией
\newcommand{\centreq}[1]{
\begin{equation}
#1
\end{equation}
}

% Главы с центрированием
\newcommand{\csection}[1]{
	\section{\centering#1}
}
\newcommand{\ocsection}[1]{
	\section*{\centering#1}
}
\newcommand{\csubsection}[1]{
	\subsection{\centering#1}
}
\newcommand{\ocsubsection}[1]{
	\subsection*{\centering#1}
}



%%%%%%%%%%%%%%%%%%%%%%%%%%%%%%%%%%%%%%

%%%%%%%%%%%%%%%%%%%%%%%%%%%%%%%%%%%%%%%%%%%



%%%%%%%%%%%% Содержание %%%%%%%%%%%%%%%%%%
\newpage
\tableofcontents
\newpage
%%%%%%%%%%%%%%%%%%%%%%%%%%%%%%%%%%%%%%%%%%



%%%%%%%%%%%% Основной текст %%%%%%%%%%%%%%

\ocsection{Введение}
Написать курсовую/диплом/отчет и разобраться в работе \LaTeX\ .
\addcontentsline{toc}{section}{Введение}

%%%%%%%%%%%%%%%%%%%%%%%%%%%%%%%%%%%%%%%%%%%%%%%%%%%%%%%%%%%%%%%%%%%%
\csection{Дерево Иерархии}
Что же такое дерево устройства? Дерево - это набор конфигурационных
файлов, необходимых для определения специфичных для конкретного устройства параметров, пакетов (приложений) и зависимостей. Как правило, дерево устройства состоит из папок \textbf{device, vendor, kernel}.
\begin{itemize}
\item В папке \textbf{device} присутствуют основные конфигурационные файлы, драйвера с открытым кодом.
\item В папке \textbf{vendor} - проприетарные файлы (бинарные файлы с закрытым исходным кодом).
\item В папке \textbf{kernel} – исходники ядра устройства.
\end{itemize}


%%%%%%%%%%%%%%%%%%%%%%%%%%%%%%%%%%%%%%%%%%%%%%%%%%%%%%%%%%%%%%%%%%%%

\csection{Основной текст}
Соображения высшего порядка, а также дальнейшее развитие различных форм деятельности требует от нас анализа модели развития. Равным образом выбранный нами инновационный путь обеспечивает широкому кругу специалистов участие в формировании новых предложений?\\
Равным образом рамки и место обучения кадров требует определения и уточнения существующих финансовых и административных условий.
Дорогие друзья, социально-экономическое развитие требует от нас анализа экономической целесообразности принимаемых решений? \cite{kistyakovskii} % здесь \citep используется для вставки цитирования в скобках


\csection{Рисунки}


Можно вручную вставлять рисунки прописывая каждый параметр



\begin{figure}[!htb]
	\centering
	\includegraphics[width=\textwidth]{image1.jpg}
	\caption{Используйте Rich Text}
	\label{fig:image1}
\end{figure}

Параметр width задаёт ширину рисунка. В этом случае она равна ширине текста (textwidth). Перед textwidth можно указать значение от 0.1 до 1.
А можно использовать кастомную команду image:

% \image {Имя изображения.расширение}{Подпись к рисунку}

\image{image2.jpg}{Подпись к рисунку}{0.7}
Она позволяет удобно и быстро вставлять и регулировать размер изображения, для того чтобы исключить проблемы с разметкой изображений на странице.


\image{image2.jpg}{То же изображение, но меньше}{0.35}


К тому же команда автоматически проставляет нумерацию рисунков.
\csection{Таблицы}
% К сожалению таблицы это комплексная вещь, поэтому придется задавать все параметры вручную

\begin{table}[h!]
	\caption{Пример работы с таблицей}
	\label{table:1}
	\centering
	\begin{tabular}{|c|c|c|c|}
	 \hline
	 Столбец1 & Столбец2 & Столбец3 & Столбец4 \\ [0.5ex]
	 \hline
	 \multirow{3}{5em}{Несколько строк} & 6 & 87837 & 787 \\
	  &  7 & 78 & 5415 \\
	   & 545 & 778 & 7507 \\
	   & 545 & 18744 & 7560 \\
	   & 88 & 788 & 6344 \\ [1ex]
	 \hline
	\end{tabular}
\end{table}


Больше о таблицах \href{https://www.overleaf.com/learn/latex/Tables}{тут}.

%%%%%%%%%%%%%%%%%%%%%%%%%%%%%%%%%%%%%%%%%%%%%%%%%%%%%%%%%%%%%%%%%%%%
\csection{Математика}

    \csubsection{Математические формулы}
    Хорошо известная теорема Пифагора \(x^2 + y^2 = z^2\) была
    доказана недействительной для других показателей.
    Это означает, что следующее уравнение не имеет целочисленных решений:
    \[ x^n + y^n = z^n \]
    Другой способ вставить уравнение в текст такой: $x^2 + y^2 = z^2$. То есть уравнение нужно поместить между двумя знаками "доллара".

		\centreq{
		G/Ker_f \cong Im_f
		}

		Гомоморфный образ группы, до победы коммунизма, изоморфен факторгруппе по ядру гомоморфизма!

		\centreq{2+2=5}

		Litteraly 1984
		\footnote{
		Крч Оурэлл сидит такой и думает "а че будет если взять "Мы" Замятина и выкинуть
		оттуда все размышления о матеше и мире)0". Кстати Оруэлл сам был социалистом, а в его романах он ругал именно сталинизм. Веселый чел был
		}

    \csubsection{Дроби}
    При отображении дробей в строке, например \(\frac{3x}{2}\),
    вы можете установить другой стиль отображения:
    \( \displaystyle \frac{3x}{2} \).
    Это также верно и в обратном направлении
    \[ f(x)=\frac{P(x)}{Q(x)} \ \ \textrm{и}
    \ \ f(x)=\textstyle\frac{P(x)}{Q(x)} \]

    \csubsection{Интегралы}
    Интеграл \(\int_{a}^{b} x^2 dx\) внутри текста.
    \medskip
    Тот же интеграл на дисплее:
    \[
    \int_{a}^{b} x^2 \,dx
    \]
    Официальнный туториал по интегралам можно посмотреть по этой  \href{https://www.overleaf.com/learn/latex/Integrals,_sums_and_limits#Integrals}{ссылке}.

    \csubsection{Сумма и произведение}
    Тоже оставлю \href{https://www.overleaf.com/learn/latex/Integrals,_sums_and_limits#Sums_and_products}{ссылку}.

    \csubsection{Пределы}

    Предел \(\lim_{x\to\infty} f(x)\) внутри текста.
    Тот же предел на дисплее:
    \[
    \lim_{x\to\infty} f(x)
    \]

%%%%%%%%%%%%%%%%%%%%%%%%%%%%%%%%%%%%%%%%%%%%%%%%%%%%%%%%%%%%%%%%%%%%
\csection{Символы}
$\alpha A$ - греческие символы,  $ \lambda; \Lambda$ - физические величины, $\exists; \forall$ - логические символы\\
По этой   \href{https://www.overleaf.com/learn/latex/List_of_Greek_letters_and_math_symbols}{ссылке} можно посмотреть остальные символы. \href{https://www.overleaf.com/learn/latex/Operators}{Здесь} - математические операторы.



%%%%%%%%%%%%%%%%%%%%%%%%%%%%%%%%%%%%%%%%%%%%%%%%%%%%%%%%%%%%%%%%%%%%
\csection{Руководство}
\href{https://www.texlive.info/CTAN/info/lshort/russian/lshortru.pdf}{Ссылка} на полное введение в Latex на русском языке.


%%%%%%%%%%%%%%%%%%%%%%%%%%%%%%%%%%%%%%%%%%%%%%%%%%%%%%%%%%%%%%%%%%%%
\newpage
\csection{Заключение}
Вот и закончили написание курсовой/диплома/отчета



\newpage
%%%%%%%%%%%%%%%%%%%%%%%%%%%%%%%%%%%%%%%%%%


%%%%%%%%%%%% Источники %%%%%%%%%%%%%%%%%%
% refs.blib
\newpage
\renewcommand*\refname{Список приложений}
\nocite{*}
\bibliography{refs}
\addcontentsline{toc}{section}{Список приложений}
%%%%%%%%%%%%%%%%%%%%%%%%%%%%%%%%%%%%%%%%%


\end{document}


%%%%%%%%%%%%%%%%%%%%%%%%%%%%%%%%%%%%%%%%%%

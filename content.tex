%%%%%%%%%%%%%%%%%%%%%%%%%%%%%%%%%%%%%%%%%%%%%%%%%%%%%%%%%%%%%%%%%%%%%%%%%%%%%%%%%%%%%%%%%%

\ocsection{Введение}
Написать курсовую/диплом/отчет и разобраться в работе \LaTeX\ .
\addcontentsline{toc}{section}{Введение}

%%%%%%%%%%%%%%%%%%%%%%%%%%%%%%%%%%%%%%%%%%%%%%%%%%%%%%%%%%%%%%%%%%%%
\csection{Дерево Иерархии}
Что же такое дерево устройства? Дерево - это набор конфигурационных
файлов, необходимых для определения специфичных для конкретного устройства параметров, пакетов (приложений) и зависимостей. Как правило, дерево устройства состоит из папок \textbf{device, vendor, kernel}.
\begin{itemize}
\item В папке \textbf{device} присутствуют основные конфигурационные файлы, драйвера с открытым кодом.
\item В папке \textbf{vendor} - проприетарные файлы (бинарные файлы с закрытым исходным кодом).
\item В папке \textbf{kernel} – исходники ядра устройства.
\end{itemize}


%%%%%%%%%%%%%%%%%%%%%%%%%%%%%%%%%%%%%%%%%%%%%%%%%%%%%%%%%%%%%%%%%%%%

\csection{Основной текст}
Соображения высшего порядка, а также дальнейшее развитие различных форм деятельности требует от нас анализа модели развития. Равным образом выбранный нами инновационный путь обеспечивает широкому кругу специалистов участие в формировании новых предложений?\\
Равным образом рамки и место обучения кадров требует определения и уточнения существующих финансовых и административных условий.
Дорогие друзья, социально-экономическое развитие требует от нас анализа экономической целесообразности принимаемых решений? \cite{kistyakovskii} % здесь \citep используется для вставки цитирования в скобках


\csection{Рисунки}


Можно вручную вставлять рисунки прописывая каждый параметр

\begin{figure}[!htb]
	\centering
	\includegraphics[width=\textwidth]{image1.jpg}
	\caption{Используйте Rich Text}
	\label{fig:image1}
\end{figure}

Параметр width задаёт ширину рисунка. В этом случае она равна ширине текста (textwidth). Перед textwidth можно указать значение от 0.1 до 1.
А можно использовать кастомную команду image:

% \image {Имя изображения.расширение}{Подпись к рисунку}

\image{image2.jpg}{Подпись к рисунку}{0.7}
Она позволяет удобно и быстро вставлять и регулировать размер изображения, для того чтобы исключить проблемы с разметкой изображений на странице.


\image{image2.jpg}{То же изображение, но меньше}{0.35}

К тому же команда автоматически проставляет нумерацию рисунков.
\csection{Таблицы}
% К сожалению таблицы это комплексная вещь, поэтому придется задавать все параметры вручную

\begin{table}[h!]
	\caption{Пример работы с таблицей}
	\label{table:1}
	\centering
	\begin{tabular}{|c|c|c|c|}
	 \hline
	 Столбец1 & Столбец2 & Столбец3 & Столбец4 \\ [0.5ex]
	 \hline
	 \multirow{3}{5em}{Несколько строк} & 6 & 87837 & 787 \\
	  &  7 & 78 & 5415 \\
	   & 545 & 778 & 7507 \\
	   & 545 & 18744 & 7560 \\
	   & 88 & 788 & 6344 \\ [1ex]
	 \hline
	\end{tabular}
\end{table}


Больше о таблицах \href{https://www.overleaf.com/learn/latex/Tables}{тут}.

%%%%%%%%%%%%%%%%%%%%%%%%%%%%%%%%%%%%%%%%%%%%%%%%%%%%%%%%%%%%%%%%%%%%
\csection{Математика}

    \csubsection{Математические формулы}
    Хорошо известная теорема Пифагора \(x^2 + y^2 = z^2\) была
    доказана недействительной для других показателей.
    Это означает, что следующее уравнение не имеет целочисленных решений:
    \[ x^n + y^n = z^n \]
    Другой способ вставить уравнение в текст такой: $x^2 + y^2 = z^2$. То есть уравнение нужно поместить между двумя знаками "доллара".

		\centreq{
		G/Ker_f \cong Im_f
		}

		Гомоморфный образ группы, до победы коммунизма, изоморфен факторгруппе по ядру гомоморфизма!

		\centreq{2+2=5}

		Litteraly 1984
		\footnote{
		Крч Оурэлл сидит такой и думает "а че будет если взять "Мы" Замятина и выкинуть
		оттуда все размышления о матеше и мире)0". Кстати Оруэлл сам был социалистом, а в его романах он ругал именно сталинизм. Веселый чел был
		}

    \csubsection{Дроби}
    При отображении дробей в строке, например \(\frac{3x}{2}\),
    вы можете установить другой стиль отображения:
    \( \displaystyle \frac{3x}{2} \).
    Это также верно и в обратном направлении
    \[ f(x)=\frac{P(x)}{Q(x)} \ \ \textrm{и}
    \ \ f(x)=\textstyle\frac{P(x)}{Q(x)} \]

    \csubsection{Интегралы}
    Интеграл \(\int_{a}^{b} x^2 dx\) внутри текста.
    \medskip
    Тот же интеграл на дисплее:
    \[
    \int_{a}^{b} x^2 \,dx
    \]
    Официальнный туториал по интегралам можно посмотреть по этой  \href{https://www.overleaf.com/learn/latex/Integrals,_sums_and_limits#Integrals}{ссылке}.

    \csubsection{Сумма и произведение}
    Тоже оставлю \href{https://www.overleaf.com/learn/latex/Integrals,_sums_and_limits#Sums_and_products}{ссылку}.

    \csubsection{Пределы}

    Предел \(\lim_{x\to\infty} f(x)\) внутри текста.
    Тот же предел на дисплее:
    \[
    \lim_{x\to\infty} f(x)
    \]

%%%%%%%%%%%%%%%%%%%%%%%%%%%%%%%%%%%%%%%%%%%%%%%%%%%%%%%%%%%%%%%%%%%%
\csection{Символы}
$\alpha A$ - греческие символы,  $ \lambda; \Lambda$ - физические величины, $\exists; \forall$ - логические символы\\
По этой   \href{https://www.overleaf.com/learn/latex/List_of_Greek_letters_and_math_symbols}{ссылке} можно посмотреть остальные символы. \href{https://www.overleaf.com/learn/latex/Operators}{Здесь} - математические операторы.



%%%%%%%%%%%%%%%%%%%%%%%%%%%%%%%%%%%%%%%%%%%%%%%%%%%%%%%%%%%%%%%%%%%%
\csection{Руководство}
\href{https://www.texlive.info/CTAN/info/lshort/russian/lshortru.pdf}{Ссылка} на полное введение в Latex на русском языке.


%%%%%%%%%%%%%%%%%%%%%%%%%%%%%%%%%%%%%%%%%%%%%%%%%%%%%%%%%%%%%%%%%%%%
\newpage
\csection{Заключение}
Вот и закончили написание курсовой/диплома/отчета



\newpage
